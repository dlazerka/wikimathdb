\documentclass{article}
\usepackage{amssymb,latexsym,amsmath}

\title {WikiMathDB Idea Description}
\author{Dzmitry Lazerka}
\date{2007-11-01}


\begin{document}

\maketitle


\section{Introduction}

Idea of the WikiMathDB project was born while trying to understand a
mathematical proof while lot of theories were involved and lot of
``obvious'' facts were omitted. Then I imagined: was there any way
to store the proof so detailed, that even computer can understand
it? The more I thought about it, the more I convinced of technical
possibility to build a computer system which goal is to contain
proofs in very detailed form; so detailed, that even computer can
check it (there already are such systems, see below). So, everyone
can view at any proof in more detail, as detailed as they need. And
allow everyone to add his own theories without interference with
others. And provide possibility of different representations on one
proof for various languages and terminologies. So, the idea was born.\\

Who needs in the WikiMathDB system? What is its the purpose?

At first, for educational purposes: if a student cannot understand
something ``obvious ''or have just forgot a well-known thing, he
always can look in the WikiMathDB for explanation. Also, teacher can
add parts of his course to help students understand the course. A
lot of universities distribute synopses and/or their video
recordings, the system will be also useful to avoid mistakes and
better lectures understanding.

At second, for science purposes: one can always have the most
up-to-date version of his work, which surely does not contain
mistakes. This will take away a need in continuously distributing
and verbose explanation the work to other mathematicians. Of course,
the system will not replace science magazines and conferences, but
will appreciably help them.\\


Yet another artificial intelligence? A theorem prover?

No. The system aims to be a DataBase. There is a wide range of
computing tools, automated theorem provers. WikiMathDB is not one of
them, its responsibilities is storing and explaining, but not
evaluation. There also is a huge number of computer-aided math
libraries, WikiMathDB differs from them by containing the logical
consistency, in the way how an author understands it. The most
closest system is a Mizar project (www.mizar.org), born in 1973 and
is now containing more than 8000 math definitions and 45000
theorems. WikiMathDB mainly differs from the Mizar project by
allowing to work with usual typographic-processed article (like in
DVI format), not with an ASCII computer language. Moreover, a common
theoretical description and discussion has taken place in a QED
project (http://www-unix.mcs.anl.gov/qed), where the idea of unified
computerized mathematical knowledge repository was born.

\newpage

\section{Examples}

{\bf Example 1 (elementary)}.\\

\hrule\bigskip

{\bf Given}: $B \subset A, b \in B$, so $b \in A$.\\

\hrule\bigskip

Why ``so''? Maybe in some theories it is not true? Here you can ask
the WikiMathDB to explain you why (by, for example, ctrl-clicking on
the word ``so''). And get the answer based on definition of the
subset operator:
$$(B \subset A) \Leftrightarrow (\forall b \in B \Rightarrow b \in
A)$$ This answer is an interpretation of already inputted definition
of \textit{subset}:
$$\forall X,Y - sets \Rightarrow [X \subset Y) \Leftrightarrow
(\forall x \in X \Rightarrow x \in Y)]$$ Where $B$ can be placed on
the place of $X$, $A$ --- on the place of $Y$, and $b$ --- on the
place of $x$.
\\

{\bf Example 2}.\\

\hrule\bigskip

{\bf Given}: \textbf{Linear basis} is a \textit{linearly
independent} \textit{spanning set}.\\

\hrule\bigskip

Having read this, one may want to know, what do they mean while
talking ``linearly independent ''and ``spanning set''? By
ctrl-clicking on these terms he can get the definitions of the
terms. Here WikiMathDB can make an easy translation from English to
math: ``is\ a''$\ \rightarrow\; ``\in''$, so, this definition will
be in form:
$$
\forall b \in B \Rightarrow (b \in L \wedge b \in S)
$$
Where L is a set of linearly independent vector sets, and S --- a
set of spanning sets. Asking the system what is ``L ''gives us a
definition of a set of linearly independent vector sets.

But, you should notice that here we omitted the field in which
``basis ''is defining: subspace and vector space. Because ``basis''
will be used if form ``basis of the subspace M in vector space V''.
Omitting these details is not allowed in the WikiMathDB. The details
like that make possible to fully understand matter by computer. So,
writing a definition like for ``basis'', author has to consider
these details while adding anything to the WikiMathDB. It is
obvious, that there are a number of such details, and they slow down
process of adding new data to the system significantly, but only
this way allows to make data logically clear. And, as the Mizar
project shows, it is solvable problem.
\\
\newpage

{\bf Example 3 (more comprehensive)}.

Let's look at one solution in field of probability theory,
characteristic functions:\\

\hrule\bigskip

{\bf Given}: Random variable $\xi$ has the Couchy distribution
$$
p_{\xi}(x) = \frac{1}{\pi} \frac{a}{a^2 + x^2}.
$$

{\bf Problem}: Find the characteristic function for $\xi$.

{\bf Solution}:
$$
f_{\xi}(t) = \frac{a}{\pi} \int_{-\infty}^\infty e^{itx}
\frac{dt}{x^2 + a^2} = \frac{a}{\pi} \int_{-\infty}^\infty
\frac{cos(t x)}{x^2 + a^2}dt.
$$

Notice that $f_\xi(t)$ -- even function, so we only need to know its
values for $t>0$. Let us differentiate both sides of the last
equation, we'll get:

$$
f_{\xi}(t) = \frac{a}{\pi} \int_{-\infty}^\infty \frac{-x\ sin(t x)}
{x^2 + a^2}dt. \eqno{(1)}
$$

It is known from the math analysis that

$$
\int_{-\infty}^\infty \frac{sin(t x)}{x}dx = \pi\;\;(t>0)
\parbox{4em}{\hspace{1em}, so\hfill} a= \frac{a}{\pi}\int_{-\infty}^\infty \frac{sin(t
x)}{x}dx.\eqno{(2)}
$$

Summarizing (1) and (2), we'll get
$$
f_\xi' + a = \frac{a}{\pi}\int_{-\infty}^\infty \frac{a^2}{x^2 +
a^2}\frac{sin(t x)}{x}dx.
$$

Let us differentiate both sides by $t$. We get $f_\xi''(t) = a^2
f_\xi(t)$. So, $f_\xi(t) = c_1 e^{at} + c_2 e^{-at}$, $t>0$. Because
of $f_\xi(t)$ -- bounded function on $\mathbb{R}$, then $c_1 = 0$,
and, basing on property $f_\xi(0) = 1$, get $c_2 = 1$. So, while $t
> 0 \; $ $\; f_\xi(t) = e^{-at}$, but taking consideration of the function
eveness, finally get $f_\xi(t) = e^{-a|t|}$, $t \in \mathbb{R}$.\\

\hrule\bigskip

Here we see a lot of hidden things: starting from definition of
characteristic function and ending with differentiation of improper
parametrized integral by parameter, along with on-the-fly solve of
Euler's differential equation. There are wasting a reader's time if
the reader wants to check (or understand) the solution. It is good
if reader is always a genius, but if he isn't? With the help of
WikiMathDB, author can waste his own time but save readers' time.
Also, explaining something in detail to the stupid computer can
eliminate all mistakes that human eye may not notice.

\newpage

\section{Global Design}

Version 1.0 should provide basic functionality for {\bf explaining}
and {\bf viewing} existent content, along with {\bf adding} new
content. These include:

\begin{enumerate}
\item
{\bf Common}. All development must use the portable technologies and
tools for compatibility, easier maintaining and development.
\item
{\bf Client side}. Basic client should be {\bf crossplatform}, has a
{\bf WYSIWYG math editor}. Added documents should be sent and stored
too, not only parsed and transformed versions. A web-application
form sounds to be a good idea.
\item
{\bf Server side}. Server side should provide an open, documented
API, to enable creating different clients for the same servers. The
SOAP technology sounds to be a good idea. Structure should consider
further features described below for future upgrades.
\end{enumerate}

Furter versions will point on:
\begin{itemize}
\item {\bf internationalization} -- distinguish different
languages, along with different terminologies for the same fields.
\item {\bf distributed server side} -- allow anyone to install their own
clones of the database, work with them, and send changes to our
repository.
\item {\bf versioning} -- track changes made to the database,
allowing reverting changes made. Forking and merging will be a good
point too.
\item {\bf authorization} -- close some parts of database from public
edit.
\item {\bf heuristics} -- help editors to add new content by
performing evaluating and computing like it done by a lot of
computing environments like Mathematica$^\circledR$.

\end{itemize}

\end{document}
